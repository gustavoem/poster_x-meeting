\include{template}
%%%%%%%%%%%%%%%%%%%%%%%%%%%%%%%%%%%%%%%%%%%%%%%%%%%%%%%%%%%%%%%%%%%%%%%%%%%%%%%%%%%%%%
 
\title{\LARGE%
A global feature selection algorithm for the\\ model 
selection step in the identification \\ of cell 
signaling networks%
}

\author{\vspace*{1.5cm}\underline{Gustavo Estrela$^{1,2,3}$}, 
                       Lulu Wu$^{1,2}$, Vincent No\"el$^{1,3}$,
                       Carlos Eduardo Ferreira$^2$,
                       Hugo A. Armelin$^{1,3}$, \\
                       Marco Dimas Gubitoso$^2$,
                       Junior Barrera$^{1,2}$,
                       and Marcelo S. Reis$^{1}$}

\institute{%
$^1$Center of Toxins, Immune-response and Cell Signaling (CeTICS), Instituto Butantan, Brazil\\%
$^2$Instituto de Matem\'atica e Estat\'istica, Universidade de S\~ao Paulo, Brazil\\%
$^3$Laboratório Especial de Ciclo Celular (LECC), Instituto Butantan, Brazil\\%
$^4$Instituto de Qu\'imica, Universidade de S\~ao Paulo, Brazil%
}


%%%%%%%%%%%%%%%%%%%%%%%%%%%%%%%%%%%%%%%%%%%%%%%%%%%%%%%%%%%%%%%%%%%%%%%%%%%%%%%%%%%%%%
\begin{document}
\begin{frame}
\begin{columns}

\leftcolumn{ 
\begin{block}{Motivation}%
\paragraph{In the context of machine learning, the feature selection 
problem consists in choosing a subset of features that best explains the 
classification with minimum redundancy. The space of solutions of this
problem induces a boolean lattice and the cost function commonly 
describes U shaped curves on chains of this lattice, what is explained 
by the growth of estimation errors as we add more features.}
\paragraph{The U shaped curves justifies the reduction to the U-Curve
problem: a special case of the feature selection problem where every 
chain of the search space describes U shaped curves. Many algorithms 
in the literacy exploit this reduction and yet they show limitations 
regarding scalability. That shows the need for new approaches on 
solving the U-Curve problem. To this end, we developed the Parallel 
U-Curve Search (PUCS).}
\end{block}


%%%%%%%%%%%%%%%%%%%%%%%%%%%%%%%%%%%%%%%%%%%%%%%%%%%%%%%%%%%%%%%%%%%%%%%%                                     
\begin{block}{Method}%
    \begin{figure}[h]
    \begin{tabular}{l r}
    \centering
    \subfigure {
        \label{fig:example:full}
        \includegraphics[clip=true, width=0.3\textwidth]{simulation/Boolean_lattice.pdf}
    }
    &
    \subfigure {
        \label{fig:example:A}
        \includegraphics[clip=true, width=0.3\textwidth]{simulation/Outer_Boolean_lattice_A.pdf}
    }
    \vspace*{.5cm} \\
    \subfigure {
        \label{fig:example:B}
        \includegraphics[ clip=true, width=0.30\textwidth]{simulation/Outer_Boolean_lattice_B.pdf}
    }
    &
    \subfigure {
        \label{fig:example:C}
        \includegraphics[clip=true, width=0.30\textwidth]{simulation/Outer_Boolean_lattice_C.pdf}
    }
    \vspace*{.5cm} \\
    \subfigure {
        \label{fig:example:D}
        \includegraphics[ clip=true, width=0.30\textwidth]{simulation/Outer_Boolean_lattice_D.pdf}
    }
    &
    \subfigure {
        \label{fig:example:E}
        \includegraphics[clip=true, width=0.30\textwidth]{simulation/Outer_Boolean_lattice_E.pdf}
    }
    \vspace*{.5cm} \\
    \subfigure {
        \label{fig:example:F}
        \includegraphics[ clip=true, width=0.30\textwidth]{simulation/Outer_Boolean_lattice_F.pdf}
    }
    &
    \subfigure {
        \label{fig:example:G}
        \includegraphics[clip=true, width=0.30\textwidth]{simulation/Outer_Boolean_lattice_G.pdf}
    }
    \vspace*{.5cm} \\
    \subfigure {
        \label{fig:example:H}
        \includegraphics[ clip=true, width=0.30\textwidth]{simulation/Outer_Boolean_lattice_H.pdf}
    }
    &
    \subfigure {
        \label{fig:example:I}
        \includegraphics[clip=true, width=0.30\textwidth]{simulation/Outer_Boolean_lattice_I.pdf}
    }
    \end{tabular}   
    \caption{PUCS dynamics on an instance of the U-Curve problem.}
    \label{fig:algorithm_toy_example} 
\end{figure}

\end{block}

\vfill 
\vspace*{-.5cm}%
\begin{block}{Acknowledgements}%
% \insidecolumns{0.2}{2}%
% {\mycenteredimage{institutions/FAPESP.jpg}{1.1}}%
% {\mycenteredimage{institutions/capes.jpg}{1.2}}%
% {\mycenteredimage{institutions/CNPq.png}{1.5}}%

\vspace*{-1.5cm}%
\begin{figure}[h]
    \begin{tabular*}{0.7\textwidth}{c@{\extracolsep{\fill}}cc}
    \centering
    \subfigure {
        \includegraphics[clip=true, width=0.2\textwidth]{institutions/FAPESP.jpg}
    }
    &
    \subfigure {
        \includegraphics[clip=true, width=0.2\textwidth]{institutions/CNPq.png}
    }
    &
    \subfigure {
        \includegraphics[clip=true, width=0.1\textwidth]{institutions/capes.jpg}
    }
    \end{tabular*}   
\end{figure}
\vspace*{1.5cm}%
\end{block}%
} % end \leftcolumn




\rightcolumn{              
\begin{block}{Results}%
\rightfigparagraph{figures/aproximation_like.png}{19}{15}{%
We implemented the PUCS algorithm on C++ language and used OpenMP
to parallelize the code. Using a server with 64 cores and 256 gigabytes
of memory we were able to confirm our expectations that the algorithm
can find solutions as good as it possible (be optimal) as long as we
increase the parameters {\em p} and {\em l}.
}%

\leftfigparagraph{qrcode_featsel.png}{4}{10}{
We used the {\em featsel} framework to benchmark PUCS with other algorithms, such as Exhaustive Search (ES), Sequential Forward Selection (SFS) and 
Backward Feature Selection (BFS).
} % paragraph

\begin{figure}[h]
    \begin{tabular}{l r}
    \centering
    \subfigure {
        \label{fig:art_res:small:time}
        \includegraphics[clip=true, width=0.45\textwidth]{results/small_artificial_time.png}
    }
    &
    \subfigure {
        \label{fig:art_res:small:correctness}
        \includegraphics[clip=true, width=0.45\textwidth]{results/small_artificial_corr.png}
    } \\
    \subfigure {
        \label{fig:art_res:big:time}
        \includegraphics[clip=true, width=0.45\textwidth]{results/big_artificial_time.png}
    }
    &
    \subfigure {
        \label{fig:art_res:big:correctness}
        \includegraphics[clip=true, width=0.45\textwidth]{results/big_artificial_corr.png}
    }
    \end{tabular}   
    \caption{Average execution time and average number of times each
    algorithm found the best solution. For these instances we used SFS 
    as a base for the PUCS algorithm.}
    \label{fig:art_res} 
\end{figure}

\begin{figure}[h]
\centering
\includegraphics[clip=true, width=1\textwidth]{results/easy_datasets.png}
\caption{Average cross-validation error when using the features selected by
    PUCS to project a support vector machine learning model.}
\label{fig:svm_error}
\end{figure}

\end{block}



%%%%%%%%%%%%%%%%%%%%%%%%%%%%%%%%%%%%%%%%%%%%%%%%%%%%%%%%%%%%%%%%%%%
\begin{block}{Conclusion}%
{

\paragraph{On our implementation of PUCS we could see that this 
algorithm can have better time performance than optimal solvers
(such as ES) and also can find better solutions than heuristics 
(such as SFS and BFS). Future and ongoing work on this project includes:
\begin{itemize}
    \item{design of poset forest-based algorithms;}
    \item{applications on cell signaling pathways.}
\end{itemize}
}}
\end{block}
}% end of right column
\end{columns}
\end{frame}
\end{document}
